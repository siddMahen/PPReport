\documentclass[12pt, a4paper, draft]{report}
\usepackage{palatino}
\usepackage{parskip}
\usepackage[titles]{tocloft}
\usepackage{amsmath}
\usepackage{alltt}
\usepackage{algpseudocode}

\begin{document}

% -- Title --
\title{An Exploration of Modern Cryptography}
\author{Siddharth\\
        Mahendraker}
\maketitle

% -- Abstract --
\begin{abstract}
Fusce dapibus, tellus ac cursus commodo, tortor mauris condimentum nibh,
ut fermentum massa justo sit amet risus. Maecenas sed diam eget risus
varius blandit sit amet non magna. Duis mollis, est non commodo luctus,
nisi erat porttitor ligula, eget lacinia odio sem nec elit. Vivamus
sagittis lacus vel augue laoreet rutrum faucibus dolor auctor. Aenean
eu leo quam. Pellentesque ornare sem lacinia quam venenatis vestibulum.
Nulla vitae elit libero, a pharetra augue. Donec id elit non mi porta
gravida at eget metus.
\end{abstract}

% -- Bookeeping
\setcounter{secnumdepth}{3}
%\setcounter{tocdepth}{3}
\renewcommand*\thesection{\arabic{section}}
\renewcommand{\cftsecfont}{\bfseries}

% -- ToC --
\setcounter{page}{1}
\pagenumbering{roman}
\tableofcontents
\clearpage
\addcontentsline{toc}{section}{Introduction}
\setcounter{page}{1}
\pagenumbering{arabic}

% -- Intro --
\section*{Introduction}

Suppose two people, Alice and Bob, wish to communicate by mail and do not
want their mailwoman, Eve, to be able to read their messages. Alice and Bob
are military personal of the same country, but they have never met each
other before. Because Eve is the mailwoman, she will be able to read
all of the messages passing between Alice and Bob, but her obligation to
the postal service prevents her from tampering with these
messages\footnotemark.

\footnotetext[1]{ In reality, Eve would not be bound by such a petty
obligation, however, for the sake of simplicity, let us assume this is
true. }

The question is, is it possible for Alice and Bob to communicate securly
in these circumstances? Astoundingly, the answer is yes!

Cryptography is the science of securly sending messages over insecure
channels. Using cryptographic techniques, Alice and Bob can be sure that
their communications are illegible to Eve.

\subsection{Terminology and Basic Concepts}

\subsubsection{Alice and Bob}
Unless specified otherwise, Alice and Bob are two parties attempting to
communicate over an insecure channel, and Eve is their adversary trying to
read their messages.

\subsubsection{Encryption and Decryption}

Messages in cryptography are formally called plaintext. When messages are
scrambled, or made ``illegible'', they are encrypted. The encrypted form
of these messages is called ciphertext. The reverse process of encryption,
decryption, accepts ciphertext as input and returns plaintext. We can
describe this mathematically as:
\begin{align*}
    E(M) & = C\\
    E'(C) & = D(C) = M
\end{align*}
Where $M$ denotes plaintext, $C$ denotes cipher text, $E$ is the encryption
function and $D$ is the decryption function, or the inverse of $E$. Also
note the following identity:
\begin{align*}
    D(E(M)) & = M\\
    E(D(C)) & = C
\end{align*}

\subsubsection{Ciphers and Keys}

A cryptographic algorithm, or cipher, is a function used for encryption
and decryption.

If the workings of a cipher are made public, then the messages
of anyone who is known to use the cipher can quickly be comprimised
by simply implementing an inverse of the cipher. Therefore, cryptographers
introduced a key. A key is a piece of secret, private information
upon which the cipher depends. It often takes the form of a number.
The total number of possible values a key can take on is called they
keyspace. Because ciphers depend on the key to encrypt and decrypt
plaintext, encryption and decryption is often denoted:
\begin{align*}
    E_K(P) & = C\\
    D_K(C) & = P
\end{align*}
The key is denoted by $K$ and the keyspace is by $\mathcal{K}$. Note that
the identity mentioned in 0.1.2 still holds true for ciphers.

\subsubsection{Symmetric and Asymmetric Ciphers}

There are two distict kinds of ciphers, symmetric ciphers and public-key
(or asymmetric) ciphers.

Symmetric ciphers are ciphers for which the key used to decrypt ciphertext
and encrypt plaintext is the same. In most symmetric ciphers, this means
that Alice and Bob will need to agree on a key before they can begin
sending messages. Symmetric ciphers can be further categorized as stream
ciphers or block ciphers. Stream ciphers operate on only one bit (or byte)
of plaintext at a time, where as block ciphers operate on a large number
of bytes at once.

Public-key ciphers are ciphers for which the the key used for encrypting
plaintext is different from the the key used for decrypting ciphertext.
Further, these keys should be independant of each other, meaning the
decryption key can not be calculated\footnotemark from the encryption key.
The design of this cipher is such that the ecryption key can be published
for anyone to use, but only the owner of the decryption key can decrypt
the message. This is why is encryption key is referred to as the public
key and the decryption key is referred to as the private key.

\footnotetext[2]{ In a reasonably amount of time ofcourse. }

\subsubsection{Cryptanalysis}

Crytanalysis is the study of obtaining the plaintext from encrypted
messages without the knowledge of the key. An attempt to cryptanalyse
a cipher is called an attack. Succeful attacks often reveal either the
plaintext, the secret key, or both.

The only assumption made in cryptanalysis is that the only piece of
information the users of the cipher, Alice and Bob know that the adversary
Eve does not is the secret key. This means that all other information,
including communications and the workings of their cryptographic algorithm
are available to anyone. This assumption implies that the security of
the algorithm rests only in the key, and nothing else.

There are three main cryptanalysis techniques we will be focusing on in
this report. Listed in decreasing order of difficulty they are; ciphertext
only attacks, known-plaintext attacks and chosen plaintext attacks.

In ciphertext only attacks, the cryptanalyst (or attacker) Eve has access
to several different ciphertexts. The attack is considered succesful if
Eve succesfully retrieves the plaintexts corresponding to the ciphertexts
or the key used in encryption.

In known-plaintext attacks, Eve has access to the ciphertexts as well
as their corresponding plaintexts. The attack is succesful if Eve
finds the key (or keys) used to encrypt each plaintext.

In chosen plaintext attacks, Eve can not only access the ciphertexts,
and their corresponding plaintext, but can also choose which plaintexts
are encrypted and which are decrypted. The attack is succesful if Eve
retrieves the key (or keys) used to encrypt each plaintext.

Note that in all of the cases above, the adversary, Eve, had to know
some amount of ``information'' about the ciphertext, plaintext or the
relationship between the two. The only other technique which can yield
the key is a brute force attack or exhaustive search attack, in which
Eve checks the ciphertext against all possible keys in the keyspace until
one of the keys reveals the plaintext.

% -- Perhaps explain how ciphers can be broken, ergo kinds of attacks
% -- Perhaps talk about keyspace, ciphertext space and plaintext space?
% -- More stuff will be introduced as needed...

% --Caesar Cipher--
\section{Substitution Ciphers}

% -- Explain what a substitution cipher is

A substitution cipher is a cipher in which each character or byte in the
plaintext is substituted with a character or byte in the cipher text.

This can be seen mathematically as:
\begin{align*}
    m & = fromChar(i)\\
    m + k & = n\\
    c & = toChar(n)
\end{align*}
Decryption works the other way around. An already shifted character, $i$,
is tranformed into it's interger representation, $m$, from which the key
is subtracted to yield the interger $n$, which is finally transformed back
into the plaintext.
\begin{align*}
    m & = fromChar(i)\\
    m - k & = n\\
    p & = toChar(n)
\end{align*}
This can be more concesily explained in pseudocode:

% -- Show pseudo code here

\subsection{Information Theory and Languages}

Before we begin a deconstruction of the Caesar cipher, there are a few
assumptions we make that must be explained.

Firstly, we must clarify the definition of information we used in section
0.X.X. Information can be rigourouly defined as the least number of bits
it would take to represent all possible meanings of a message, assuming
all messages are equally likely.

For example, suppose we are trying to determine the amount of information
in a list of possible sexes:

\begin{center}
\begin{verbatim}
1. Male
2. Female
\end{verbatim}
\end{center}

Clearly, this data can be represented using one bit, where the 1
represents male and 0 represents female. Therefore, we can say that there
is only one bit of information present in this list.

Now, if we take a look at words used in the English language, we clearly
see that English does not represent this information very succintly, e.i
there is lots of redundance per character.

For example, the sentance ``met u tmrw @ 9'' conveys the same information
as the sentance ``meet you tomorrow at nine'', yet does do much more
succintly. Therefore, we could say that many of the character in the
latter sentance are redundant or useless.

Although this may not seem related at all to cryptography, it is. This
redundancy in languages causes sentances to ``leak'' more information than
they need to. As we shall soon see, this often manifests itself as
discrepancies in the frequency and location of certain characters in
relation to others, and makes breaking the Caesar cipher a piece of
cake.

% -- I should create footnotes where you can read more about
% -- these subjects, because my explanations are less than adequate

\subsection{Cryptanalysis of the Caesar Cipher}

If we take the most naïve cryptanalytic approache, a brute force attack,
the Caesar cipher appears quite strong. Indeed the keyspace of the cipher
is $26!$ or approximately $4 \times 10^{26}$! This means that even if we
were to check even a million keys per second, it would still take us around
$1.27 \times 10^{13}$ years to check every possible key! That longer than
the estimated age of the universe!!

However, we know from our understanding of redundancy in language that there
is information being leaked here.

Because the output of this algorithm mearely ``switches'' one letter with another,
the letters in any particular ciphertext will continue to follow the known
statistic rules regarding English text. Particularly, they will maintain certain
distributions of characters, bi-grams and tri-grams over the message.

\begin{table}
\begin{tabular}{| c | c |}
    \hline
    \multicolumn{2}{| c |}{By decreasing frequency} \\
    \hline
    E & 13.11\% \\
    T & 10.47\% \\
    A & 8.15\% \\
    O & 8.05\% \\
    N & 7.15\% \\
    R & 6.85\% \\
    I & 6.35\% \\
    S & 6.15\% \\
    H & 5.25\% \\
    D & 3.75\% \\
    L & 3.35\% \\
    F & 2.95\% \\
    C & 2.75\% \\
    M & 2.55\% \\
    U & 2.45\% \\
    G & 1.95\% \\
    Y & 1.95\% \\
    P & 1.95\% \\
    W & 1.55\% \\
    B & 1.45\% \\
    V & 0.95\% \\
    K & 0.45\% \\
    X & 0.15\% \\
    J & 0.15\% \\
    Q & 0.15\% \\
    Z & 0.05\% \\
    \hline
\end{tabular}
\end{table}

% -- Also include tables with bi-grams and tri-grams

Therefore, if we are given the following ciphertext:

% -- TODO: remember that for alltt blocks, lines should
% --       end around 53~

\begin{alltt}
ofobiyxocryevnkvcyexnobcdkxndrovswsdkdsyxcypmbizdyqbkz
rikckdyyvgroxeconsxmyxtexmdsyxgsdrcyvsnzbyqbkwwsxqzbkm
dsmockxnpsbwwkdrowkdsmkvmyxtomdebocsdmkxlorsqrvioppomd
sforygofobspwscecondrobowkilonsckcdobyecmyxcoaeoxmocsx
mvensxqwkccnkdkdropdybcobfobrsqrtkmusxq
\end{alltt}

We would first construct a table of the character present in the text and
their respective frequencies, as so:

% -- Freq. Table

Then we would map the most frequent characters to each other.
Clearly, the character ``o'' appears to represent the character ``e''.
Based on our knowledge of the algorithm, we know the key is used as
a shift for the integer values of each character. A quick glance at the
ASCII character table reveals that e = 101 and o = 111 as integers.
Therefore, we can conclude that the key used to encrypt this plaintext
is the number 10.

A quick check reveals that we were indeed correct.

\begin{alltt}
everyoneshouldalsounderstandthelimitationsofcryptograp
hyasatoolwhenusedinconjunctionwithsolidprogrammingprac
ticesandfirmmathematicalconjecturesitcanbehighlyeffect
ivehoweverifmisusedtheremaybedisasterousconsequencesin
cludingmassdatatheftorserverhighjacking
\end{alltt}

With proper punctuation and capitalization the plaintext becomes:

\begin{alltt}
Everyone should also understand the limitations of cry
ptography as a tool. When used in conjunction with sol
id programming practices and firm mathematical conject
ures, it can be highly effective. However, if misused,
there may be disasterous consequences, including mass
data theft or server highjacking.
\end{alltt}

\subsection{Advantages and Disadvantages of the Caesar Cipher}

At this point, you may be asking yourself why the Caesar cipher would
ever be considered a viable method of encrpyting data, considering we
have been able to break it quite easily using a ciphertext only attack.

Although this cipher is very deeply flawed, it still has certain
advantages which make it practical in certain situations.

For example, if speed and memory are your main concerns, and your
plaintext only has to be superficially secure, then this cipher is one
of your best options. The Caesar cipher has a time complexity of
$\Theta(1)$ and a memory complexity of $\Theta(1)$. This means that
the cipher's speed and memory usage stay within a constant range, and
do not grow (or shrink) in relation to the size of the cipher's input.
This is because the cipher operates on only one character (or byte) at
a time, and performs the same constant time operation on each byte.

Furthermore, the Ceasar cipher may also be practical in situations where
only a small amount of plaintext is being encrpyted. The statistical
analysis which was used is only relevant to plaintexts of sufficient
length. It has been shown that highly competent cryptanalysts can break
the Caesar cipher using only 25 English characters of plaintext.
Therefore, the Caesar cipher might be practical for messages shorter than
25 characters.

%--Block Cipher--
\section{Block Ciphers}

%--ECC--
\section{ECC - Elliptic Curve Cryptography}

\end{document}
