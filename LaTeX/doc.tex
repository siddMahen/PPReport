\documentclass[12pt, a4paper, draft]{report}
\usepackage{palatino}
\usepackage{parskip}
\usepackage[titles]{tocloft}
\usepackage{amsmath}

\begin{document}

% -- Title --
\title{An Exploration of Modern Cryptography}
\author{Siddharth\\
        Mahendraker}
\maketitle

% -- Abstract --
\begin{abstract}
Fusce dapibus, tellus ac cursus commodo, tortor mauris condimentum nibh,
ut fermentum massa justo sit amet risus. Maecenas sed diam eget risus
varius blandit sit amet non magna. Duis mollis, est non commodo luctus,
nisi erat porttitor ligula, eget lacinia odio sem nec elit. Vivamus
sagittis lacus vel augue laoreet rutrum faucibus dolor auctor. Aenean
eu leo quam. Pellentesque ornare sem lacinia quam venenatis vestibulum.
Nulla vitae elit libero, a pharetra augue. Donec id elit non mi porta
gravida at eget metus.
\end{abstract}

% -- Bookeeping
\setcounter{secnumdepth}{3}
%\setcounter{tocdepth}{3}
\renewcommand*\thesection{\arabic{section}}
\renewcommand{\cftsecfont}{\bfseries}

% -- ToC --
\setcounter{page}{1}
\pagenumbering{roman}
\tableofcontents
\clearpage
\addcontentsline{toc}{section}{Introduction}
\setcounter{page}{1}
\pagenumbering{arabic}

% -- Intro --
\section*{Introduction}

Suppose two people, Alice and Bob, wish to communicate by mail and do not
want their mailwoman, Eve, to be able to read their messages. Alice and Bob
are military personal of the same country, but they have never met each
other before. Because Eve is the mailwoman, she will be able to read
all of the messages passing between Alice and Bob, but her obligation to
the postal service prevents her from tampering with these
messages\footnotemark.

\footnotetext[1]{ In reality, Eve would not be bound by such a petty
obligation, however, for the sake of simplicity, let us assume this is
true. }

The question is, is it possible for Alice and Bob to communicate securly
in these circumstances? Astoundingly, the answer is yes!

Cryptography is the science of securly sending messages over insecure
channels. Using cryptographic techniques, Alice and Bob can be sure that
their communications are illegible to Eve.

\subsection{Terminology and Basic Concepts}

\subsubsection{Alice and Bob}

Unless specified otherwise, Alice and Bob are two parties attempting to
communicate over an insecure channel, and Eve is their adversary trying to
read their messages.

\subsubsection{Encryption and Decryption}

Messages in cryptography are formally called plaintext. When messages are
scrambled, or made "illegible", they are encrypted. The encrypted form
of these messages is called ciphertext. The reverse process of encryption,
decryption, accepts ciphertext as input and returns plaintext. We can
describe this mathematically as:
\begin{align*}
    E(M) & = C\\
    E'(C) & = D(C) = M
\end{align*}
Where $M$ denotes plaintext, $C$ denotes cipher text, $E$ is the encryption
function and $D$ is the decryption function, or the inverse of $E$. Also
note the following identity:
\begin{align*}
    D(E(M)) & = M\\
    E(D(C)) & = C
\end{align*}

\subsubsection{Ciphers and Keys}

A cryptographic algorithm, or cipher, is a function used for encryption
and decryption.

If the workings of a cipher are made public, then the messages
of anyone who is known to use the cipher can quickly be comprimised
by simply implementing an inverse of the cipher. Therefore, cryptographers
introduced a key. A key is a piece of secret, private information
upon which the cipher depends. It often takes the form of a number.
The total number of possible values a key can take on is called they
keyspace. Because ciphers depend on the key to encrypt and decrypt
plaintext, encryption and decryption is often denoted:
\begin{align*}
    E_K(P) & = C\\
    D_K(C) & = P
\end{align*}
The key is denoted by $K$ and the keyspace is by $\mathcal{K}$. Note that
the identity mentioned in 0.1.2 still holds true for ciphers.

\subsubsection{Symmetric and Asymmetric Ciphers}

There are two distict kinds of ciphers, symmetric ciphers and public-key
(or asymmetric) ciphers.

Symmetric ciphers are ciphers for which the key used to decrypt ciphertext
and encrypt plaintext is the same. In most symmetric ciphers, this means
that Alice and Bob will need to agree on a key before they can begin
sending messages. Symmetric ciphers can be further categorized as stream
ciphers or block ciphers. Stream ciphers operate on only one bit (or byte)
of plaintext at a time, where as block ciphers operate on a large number
of bytes at once.

Public-key ciphers are ciphers for which the the key used for encrypting
plaintext is different from the the key used for decrypting ciphertext.
Further, these keys should be independant of each other, meaning the
decryption key can not be calculated\footnotemark from the encryption key.
The design of this cipher is such that the ecryption key can be published
for anyone to use, but only the owner of the decryption key can decrypt
the message. This is why is encryption key is referred to as the public
key and the decryption key is referred to as the private key.

\footnotetext[2]{ In a reasonably amount of time ofcourse. }



% -- More stuff will be introduced as needed...
% -- Explain cryptosystem and sym/asym

% --Caesar Cipher--
\section{Substitution Ciphers}

% -- Explain what a substitution cipher is

A substitution cipher is a cipher which

The first cipher we will be analyzing is a substitution cipher called
the Caesar cipher. In ancient times the Caesar cipher was used by Caesar
and his generals to pass along messages on the battle field. Originally,
the cipher shifted it's input 3 characters to the right. Therefore, text
such as:

\texttt{the quick brown fox jumped over the lazy dog}

becomes

\texttt{wkh txlfn eurzq ira mxpsv ryhu wkh odcb grj}.

Although this looks complex, the flaws in this cipher will soon be all
too aparent.

%--Block Cipher--
\section{Block Ciphers}

%--ECC--
\section{ECC - Elliptic Curve Cryptography}

\end{document}
